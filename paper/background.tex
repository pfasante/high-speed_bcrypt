%!TEX root = main.tex

\section{Background}
\label{sec:background}

In this section, we introduce the bcrypt algorithm and outline the
computationally expensive steps to motivate the design decisions we made. The
second part gives a short overview of our two target FPGA families and their
features.

\subsection{The bcrypt password hash}
\label{sec:bcrypt}


Provos and Mazi\`{e}res published the bcrypt hash function~\cite{BCRYPT_paper}
in 1999, which, at its core, is a cost-parameterized, modified version of the
Blowfish encryption algorithm. The key concepts are a tunable cost parameter and
the pseudo-random access of a 4 KByte memory. bcrypt is used as the default
password hash in OpenBSD since version 2.1~\cite{BCRYPT_paper}. Additionally, it
is the default password hash in current versions of Ruby on Rails and PHP.

\begin{figure*}[tb] \centering
\removelatexerror
\begin{minipage}{.45\linewidth}
\begin{algorithm2e}[H]
	\KwIn{cost, salt, key}
	\KwOut{state}
	$state \leftarrow$ InitState$()$\;
	$state \leftarrow$ ExpandKey$(state, salt, key)$\;
	\KwSty{Repeat} ($2^{cost}$) \Begin{
		$state \leftarrow$ ExpandKey$(state, 0, salt)$\;
		$state \leftarrow$ ExpandKey$(state, 0, key)$\;
	}
	\KwRet{$state$}\;
	\caption{EksBlowfishSetup}
	\label{alg:eksblowfishsetup}
\end{algorithm2e}
\end{minipage}
	\hspace{0.5cm}
\begin{minipage}{.45\linewidth}
\begin{algorithm2e}[H]
	\KwIn{cost, salt, key}
	\KwOut{hash}
	$state \leftarrow$ EksBlowfishSetup$(cost,salt,key)$\;
	$ctext \leftarrow$ ``OrpheanBeholderScryDoubt''\;
	\KwSty{Repeat} ($64$) \Begin{
		$ctext \leftarrow$ EncryptECB$(state, ctext)$\;
	}
	\KwRet{Concatenate$(cost,salt,key)$}\;
	\caption{bcrypt}
	\label{alg:bcrypt}
\end{algorithm2e}
\end{minipage}
\end{figure*}

bcrypt uses the parameters \textit{cost}, \textit{salt}, and \textit{key} as
input. The number of executed loop iterations is exponential in the
\textit{cost} parameter, \cf Algorithm~\ref{alg:eksblowfishsetup}
(\texttt{EksBlowfishSetup}). The computation is divided into two phases: First,
Algorithm~\ref{alg:eksblowfishsetup} initializes the internal state, which has
the highest impact on the total runtime. Afterwards, Algorithm~\ref{alg:bcrypt}
(\texttt{bcrypt}) encrypts a fixed value repeatedly using this state.


In its structure, bcrypt makes heavy use of the Blowfish encryption function.
This is a standard 16-round Feistel network, which uses SBoxes and subkeys
determinded by the current $state$. Its blocksize is 64-bit and during every
round, an f-function is evaluated: it uses the 32-bit input as four 8-bit
addresses for the SBoxes and computes $(S_0(a) + S_1(b)) \oplus S_2(c) + S_3(d)$
%
\texttt{EksBlowfishSetup} is a modified version of the Blowfish key schedule. It
computes a state, which consists of 18 32-bit subkeys and four SBoxes -- each
256 $\times$ 32 bits in size -- which are later used in the encryption process.
The state is initially filled with the digits of $\pi$ before an
\texttt{ExpandKey} step is performed: After adding the input key to the subkeys,
this step successively uses the current state to encrypt blocks of its
\textit{salt} parameter and updates it with the resulting ciphertext. In this
process, \texttt{ExpandKey} computes 521 Blowfish encryptions. If the salt is
fixed to zero, the function resembles the standard Blowfish key schedule. An
important detail is that the input key is only used during the very first part
of the \texttt{ExpandKey} steps. \texttt{bcrypt} finally uses
\texttt{EncryptECB}, which is effectively a Blowfish encryption.

\subsection{Special-Purpose Hardware}
\label{sec:special-hardware}

While general-purpose hardware, \ie CPUs, offers a wide variety of instructions
for all kinds of programs and algorithms, usually only a smaller subset is
important for a specific task. More importantly, the generic structure and
design of the architecture might impose restrictions and become cumbersome, \ie
when registers are too small or memory access latency becomes a bottleneck.
Reconfigurable hardware like FPGAs and special-purpose hardware like ASICs are
far more specialized -- they are dedicated to a single task.

An FPGA consists of a large area of programmable logic resources (the fabric),
\eg lookup tables (LUTs), shift registers, multiplexers and storage elements,
and a fixed amount of dedicated hardware modules, \eg memory cores (BRAM),
digital signal processing units or even processor hardcores, and can be
specialized for a given task.
%
In this work, we target two different Xilinx FPGA families. The main platform is
zedboard, more precisely its Zynq-7000 XC7Z020 FPGA. It is located in the
low-power low-cost segment. The second target is the Virtex-7 XC7VX485T FPGA
which is a high-performance device.

The Zynq-7000 consists mainly of a dual-core ARM Cortex A9 CPU, while the fabric
area and resources are comparable to an Xilinx Artix-7 FPGA. The zedboard allows
easy access to the logic inside the fabric and memory modules via direct memory
access and provides several interfaces, \eg AXI4, AXI4-Stream, AXI4-Lite or
Xillybus. It is a good choice for hardware/software co-design and in the context
of this work provides a self-contained system including complex means for
password generation and fast hardware designs. The Virtex-7 on the other hand,
offers a five times larger fabric area and seven times more memory cores at the
cost of more power consumption and a higher device price.
