%!TEX root = main.tex

\section{Conclusion and Future Work}
\label{sec:conclusion}
In this paper, we presented a highly optimized bcrypt implementation on FPGAs.
We use a quad-core structure to achieve an optimal resource utilization and gain
a speed-up of 42\% and -- due to lower power consumption -- increased
power-efficiency by 127\% compared to the previous results on the same device.
In the design we presented, the critical path is still within the Blowfish core,
resulting in a moderate clock-frequency of 100 MHz. An idea to improve this is to
pipeline the encryption within a quad-core, interleaving the computations of the
core. This may shorten the critical path further, allowing higher clock
frequencies and more parallel bcrypt cores due to shared resources.

We showed that it is possible to utilize the remaining fabric area to
implement a small on-chip password generation, which is adaptable and may be
combined with a dictionary attack, \eg for prefix and suffix modifications.
These possibilities should be evaluated and further analyzed, as the password
generation has a high impact on the success rate. Even more importantly, using
only off-chip password generation, \ie by using a CPU to generate passwords and transfer
them to the FPGA, introduces two potential bottlenecks: the software
implementation itself and the data bus. With the combination of off-chip
creation and on-chip modification, it should be possible to reduce the risk of these
bottlenecks even in large and highly parallelized clusters: We can use the password
generator construction for simple mangling rules and relax the interface or
dedicate several cores to brute-force attacks, while others work on a dictionary.
This leads to more possible trade-offs in terms of interface speed vs. area
consumption.

In our attack scenarios, we considered modern representatives of CPUs as well as
GPUs and benchmarked the (ocl)Hashcat bcrypt implementation on these platforms.
We compared the total costs of low-power and high-performance devices in two
scenarios: simple brute-force with a fixed runtime of 1 month (cost 5) and an
advanced attack with a timeframe of 1 day (cost 12). In both cases, the high
power consumption of CPUs and GPUs renders large-scale attacks infeasible, as
our FPGA implementation not only outperforms these devices but also requires
significantly less power. 

Interestingly, in combination with improved methods to derive suitable password
candidates, the overall costs when using a fast and power-efficient FPGA
implementation are not as high as expected for reasonable parameters.  As a
result, we should evaluate and adjust the parameters used in practice to
withstand the advances in technology and intelligent password generation.
